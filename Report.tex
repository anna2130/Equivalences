\documentclass{report}

\usepackage{titletoc}
\usepackage{parskip}

\makeatletter
\def\@makechapterhead#1{%
  \vspace*{50\p@}%
  {\parindent \z@ \raggedright \normalfont
    \interlinepenalty\@M
    \Huge \bfseries #1\par\nobreak
    \vskip 40\p@
  }}
  \makeatother

\begin{document} 

\title{Equivalence Teaching Tool} 
\author{Anna Thomas} 

\maketitle 

\tableofcontents 

\chapter{Introduction} 

What is the problem, why is it interesting and what’s your main idea for solving it?

We want to develop an android application for students to complete logical equivalences on. 

\section{Motivation} 
All logics are based on propositional logic in some form, so it's important that new students learn how to use it. Propositional logic consists of syntax, semantics and proof theory. Syntax is the formal lanuage that is used to express concepts, semantics provide meaning for the language and proof theory provides a way to convert one formula into another using a defined set of rules.

We know that new students learning propositional logic can struggle to understand the rules and how they should be applied to formulae. To help with this our idea is to create an equivalence teaching tool, this will be a tablet application that will allows a user to apply rules to a formula until they have reached the required equivalence.

\section{Approach}

We decided early in the project that the tool should be an android tablet application as this allows for an intuitive, interactive design while still having a large screen space.

The tool can be devided up into it's main component parts: the parser, the rule generator and tablet interface. The parser will be generated by the ANTLR4 parser generator and the rule generator and tablet interface will be written in Java using the Android SDK.

The parser will ...

The goal of the rule generator is not only to generate rules to apply to the current formula but to also determine the optimal numSber of rules to complete the equivalence. This will be used in for generating equivalences, helping the user and spotting mistakes.

The interface will have an intuitive design displaying the current formula as it's formation tree and allowing a user to click on the operators to apply a rule. This tree structure will be created by ...

\section{Contributions}

The application should have some key features, these are outlined below:

\begin{enumerate}
\item Graphical tree representation of formulae

Representing the formula as a tree structure allows a user to see exactly how the formula should be read and can help understanding of the order of operations. It also allows the user to click on an operator to select a rule for it which is much more intuative than just clicking on the whole formula and then selecting a section.

\item Undo/Redo functionality

(Add diagram)

Previous equivalent formulae will be displayed above the current formula. When an old formula is selected it will expand into tree form and the formulae below it are faded out (Undo). This allows a user to then perform rules on the old tree or select one of the later faded trees (Redo). When a rule is applied the faded formulae will then be removed from the history allowing the user to continue on from that previous point.

\item Generated equivalences

We allow the user to have equivalences automatically generated for them to solve. This is implemented by running the tool on a generated to give a significantly different equivalent one. The key advantage of having this feature as well as allowing manual entry of equivalences is that the user will have a continuous supply of new equivalent logic formulae after completing all those set by their lecturer. This also requires generating the initial formula for the tool to be run upon or to allow the user to manually enter the first formula and have the second one generated for them.

\item Difficulty setting

Extending the idea of generated equivalences we allow the user to select a difficulty. This is calculated by the length of the generated formula and how many rules were applied to get its equivalent formula. We also provide a recommended difficulty based on how many previous equivalences they have completed and how far from optimal their solutions were.

\item Help

Providing the user with help is key their improvement. Once an equivalence has been set up the tool calculates the optimal route from the start formula to the end. Upon finishing the equivalence the user will recieve a message telling them how far from the optimal solution they were and give them the option to try again or to view the optimal solution. Throughout there is also a help button available that will suggest the next step to the user on request, this includes the recommendation to undo certain steps if the user will reach a cycle. Pointing out mistakes can also be enabled, so if the user has completed a cycle or is heading towards a dead end they will be prompted that they should consider a different strategy.

\end{enumerate}


\chapter{Background}

We are assuming a basic understanding of propositional logic, including the operators and rules that are defined in the system. For more information on these please visit the Wikipedia article\cite{propositionalwiki}.

%TODO: Reference where to find more info, wikipedia? Show what we assume they know, by example?

\section{Propositional Logic}

Propositional logic is a branch of logic that studies ways of combining and modifying whole sentences, statements or propositions to form more complex propositions. It is a formal system containing logical relations and properties which are derived from these methods of joining or altering statements.

A logical system contains three major parts:

\begin{enumerate}
\item Syntax - the formal lanuage that is used to express concepts.
\item Semantics - provide meaning for the language.
\item Proof theory - provides a way to convert one formula into another using a defined set of rules.
\end{enumerate}

Logical definitions:

\begin{itemize}
\item \emph{Atomic} - A formula whose logical form is $\top$, $\bot$ or \textit{p} for an atom \textit{p}.
\item \emph{Negated atomic} - A formula of the form $\neg$\textit{p}.
\item \emph{Negated formula} - A formula of the form $\neg$\textit{A} for a formula \textit{A}.
\item \emph{Conjunction} - A formula of the form \textit{A}$\land$\textit{B}.
\item \emph{Disjuntion} - A formula of the form \textit{A}$\lor$\textit{B}.
\item \emph{Implication} - A formula of the form \textit{A}$\to$\textit{B}, where \textit{A} is the \emph{antecedent} and \textit{B} is the \emph{consequent}.
\item \emph{Literal} - A formula that is either atomic or negated atomic.
\item \emph{Clause} - A disjunction of one or more literals.
\end{itemize}

A \textit{statement} is defined as a meaningful declarative sentence that is either true or false. For example, a statement could be: 

\begin{itemize}
\item `Socrates is a man.'
\item `All men live on Earth.'
\end{itemize}
A statement can be constructed of multiple parts, for example, the above statements can be combined into:

\begin{itemize}
\item `Socrates is a man and all men live on Earth.'
\end{itemize}
Each part of this statement can be considered a proposition. Propositional logic involves studying the connectives that join these such as \textit{`and'} and \textit{`or'} (to form conjunctions and disjunctions), the rules that determine the truth values of the propositions, and what that means for the validity of the statement.

\section{Truth tables}

It is necessary to understand the meanings of the symbols used in a language. A truth table is a mathematical table used in logic to compute the functional values of logical expressions. They can be used to tell whether or not a propositional logic statement is logically valid.

A \textit{situation} determines whether each propositional atom is true or false. A truth table shows all the situations the input variables can be in. We write 1 for true and 0 for false as shown below:

\vspace{5 mm}
\begin{center}
  \begin{tabular}{ || c | c || c | c || }
    \hline
    A & B & A$\land$B & A$\lor$B \\ \hline
    1 & 1 & 1 & 1 \\
    1 & 0 & 0 & 1\\
    0 & 1 & 0 & 1 \\
    0 & 0 & 0 & 0 \\
    \hline
  \end{tabular}
\end{center}
\vspace{5 mm}

Truth tables can be used to define any operators, including any new ones you might define.

\section{Android}


\section{Related Work}


\subsection{Logic Daemon}


\subsection{Pandora}


\subsection{Propositional Logic Calculator}


\subsection{Previous Year Projects}


\chapter{Project Plan}


\chapter{Evaluation Plan}


\begin{thebibliography}{9}

\bibitem{propositionalwiki}
  Wikipedia,
  \emph{Propositional Logic}.
  http://en.wikipedia.org/wiki/Propositional\_calculus

\bibitem{propositionaliep}
  IEP,
  \emph{Propositional Logic}.
  http://www.iep.utm.edu/prop-log

\end{thebibliography}

\end{document}










